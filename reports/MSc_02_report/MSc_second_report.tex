%\section{FitzHugh-Nagumo Model Dynamics in [PAN12] and [GHO08]}
%
\documentclass[12pt]{article}
\usepackage{geometry}
\geometry{
 a4paper,
 total={210mm,297mm},
 left=40mm,
 right=20mm,
 top=35mm,
 bottom=35mm, }


\usepackage[utf8]{inputenc}
\usepackage{authblk}
\usepackage{graphicx}
\usepackage[font=small,labelfont=bf]{caption}
\usepackage{amsmath}



\title{MSc Thesis : Simulation of Brain Functional Connectivity on Emprical and Randomized Complex Networks, \\ 2nd Report}
\author[1]{\c{S}eyma Bayrak \thanks{seyma.bayrak@st.ovgu.de}}
\author[ ]{\\ Advisers: Philipp H\"{o}vel, Vesna Vuksanovi\'c}
\affil[1]{\footnotesize{Integrative Neuroscience MSc, Otto von Guericke University, Magdeburg}}



\date{May 2014}
\begin{document}
   \maketitle
   
This report presents the FitzHugh-Nagumo attractor and inhibitor dynamics used in [PAN12], [VUK13] and [GHO08] papers. 

\section{FitzHugh-Nagumo (FHN) Model in [PAN12]} 

\subsubsection{Local Dynamics}

This assignment investigates the local dynamics of the FHN model in [PAN12]. Neither of the activator and inhibitor variables includes any coupling parameter for the simplicity at the beginning. We can consider a single node having activator and inhibitor variables isolated from other nodes in the bigger network. 
   
\begin{subequations} \begin{align} \varepsilon  \dot{x} = x - \frac{x^3}{3} -y   \label{eqn: frobenius 6}\\  \dot{y} = x+a \label{eqn: frobenius 7}   \end{align} 
\end{subequations}

where $x$ is activator variable, $y$ is the inhibitor variable, $\varepsilon$ denotes the time constant accelerating $x$ , and $a$ is the threshold parameter. 


The fixed point is where $\dot{x}=0$ and $\dot{y}=0$, corresponding 
to $( -a, -a+\frac{a^3}{3} )$. The eigenvalues of Jacobian matrix are 
 $\lambda_{1,2}=\frac{1-a^2\pm \sqrt{(1-a^2)^2-4\varepsilon}}{2\varepsilon}$, $\varepsilon=0.005$, for $\lambda<1$ the fixed point is stable, for $|\lambda|>1$ unstable.


\begin{figure}[h!]
	\centering
	\includegraphics[width=\textwidth]{PAN12_local_dynamics.eps}
	\caption{The neuronal system of type-II excitability related to a Hopf bifurcation, the local dynamics with the parameters $a=1.3$ and $\varepsilon = 0.01$. The time evolution shows a rapid excitation and a weaker rapid inhibition at very beginning, over time both $x$ and $y$ stays constant. The state space figure on the right shows the pattern of $x$ and $y$ together with the nullclines. $x$ values evolve faster than $y$, the fixed point is marked with a black dot at the intersection of nullclines. Eigenvalues are $\lambda_1=-1.48$, $\lambda_2 = -67.52$, therefore the system is stable, , in other words the sytem is "excitable". }
\end{figure}

\newpage

\subsubsection{Global Dynamics with Mutual Coupling and Time Delays}

Now we can consider two nodes coupled system with mutual coupling. Each node is subject to the delayed response from the other one [PAN12]. 

\begin{subequations} \begin{align} \varepsilon  \dot{x_1} = x_1 - \frac{x_1^3}{3} -y_1  + C [x_2(t-\tau_2^C)-x_1(t)] \label{eqn: frobenius 10}\\  \dot{y_1} = x_1+a \label{eqn: frobenius 11}  \\ \varepsilon  \dot{x_2} = x_2 - \frac{x_2^3}{3} -y_2  + C [x_1(t-\tau_1^C)-x_2(t)] \label{eqn: frobenius 12}  \\  \dot{y_2} = x_2+a \label{eqn: frobenius 13} 
\end{align} 
\end{subequations}

where $C$ is the mutual coupling constant, $\tau^C$ is the time delays, and subindices $1$ and $2$ stand for the two nodes. The time delays must not be necessarily same, however, this assignment keeps both $\tau^C$ homogeneous.

\begin{figure}[h!]
	\centering
	\includegraphics[width=\textwidth]{PAN12_global_dynamics_C.eps}
		\caption{The global dynamics of the two nodes with the parameters $a=1.3$ and $\varepsilon = 0.01$, $C=0.5$ and $\tau^C= 3.0$, $x$ is the fast activator, $y$ is the slow inhibitor. The fixed point $[-1.3 , -0.56]$ is stable.}
\end{figure}

\newpage 
 The time evolution shows a rapid strong excitatory and rapid weak inhibitory behaviors of the nodes 1 and 2 as the subsequent periodic oscillations over time. The mutual coupling ($C$) let the $x$ and $y$ variables to oscillate instead of decaying onto a constant value as previously in Figure 1, the excitation happens due to mutual coupling. The phase space figures indicate a round trip over the nullclines including the fixed point.
 
\begin{figure}[h!]
	\centering
	\includegraphics[width=\textwidth, height=8cm]{PAN12_global_dynamics_C_t_small.eps}
		\caption{The global dynamics of the two nodes with the parameters $a=1.3$ and $\varepsilon = 0.01$, $C=0.5$, $\tau^C= 3.0$, $K=0$, $\tau^K= 3.0$. This figure is reproduced in order to visualize the oscillation phase shift of two nodes in detail.}
\end{figure} 
 
\subsubsection{Global Dynamics with Mutual Coupling, Self Coupling and Time Delays}

This part investigates the complete FHN model given in [PAN12].

\begin{subequations} \begin{align} \varepsilon  \dot{x_1} = x_1 - \frac{x_1^3}{3} -y_1  + C [x_2(t-\tau_2^C)-x_1(t)] + K(x_1(t-\tau_1^K) - x_1(t)) \label{eqn: frobenius 14}\\  \dot{y_1} = x_1+a \label{eqn: frobenius 15}  \\ \varepsilon  \dot{x_2} = x_2 - \frac{x_2^3}{3} -y_2  + C [x_1(t-\tau_1^C)-x_2(t)] + K(x_2(t-\tau_1^K) - x_2(t)) \label{eqn: frobenius 16}  \\  \dot{y_2} = x_2+a \label{eqn: frobenius 17} 
\end{align} 
\end{subequations}

where $K$ is the strength of self feedback of the nodes and $\tau^K$ delay are identical for the self coupling, both nodes are assumed to have the same self coupling delays for simplicity.

\begin{figure}[h!]
	\centering
	\includegraphics[width=\textwidth, height=8cm]{PAN12_global_dynamics_C_K.eps}
		\caption{The global dynamics of the two coupled neuronal oscillators with the parameters $a=1.3$ and $\varepsilon = 0.01$, $C=0.5$, $\tau^C= 3.0$, $K=0.1$, $\tau^K= 3.0$. The system is stable. }
\end{figure}

The mutual coupling constitutes the oscillatory behavior of excitation and inhibition at a higher frequency as seen in time evolution plot in Figure 4 compared to Figure 2. The phase space changes only slightly.


\section{FHN Model in [GHO08a]}

\subsubsection{Local Dynamics}

\begin{subequations}
 \begin{align}\dot{x} = \tau (y + \gamma x - \frac{x^3}{3})  \label{eqn: frobenius 17}\\  \dot{y} = -\frac{1}{\tau} (x - \alpha + b y - I ) \label{eqn: frobenius 18}   \end{align} 
\end{subequations}

where $x$ is activator variable, $y$ is the inhibitor variable, $\tau$ denotes the time constant accelerating $x$, $I$ is the external stimulus (equal to zero in order to get a transient state) and $\gamma$, $\alpha$, $b$ are the parameters. 

The eigenvalues of the Jacobian matrix is given as the following:

\begin{equation*}
\lambda_{1,2} = \dfrac{\tau^2 b (\gamma - x_f ^2) \pm \sqrt{\tau^4 b^2 (\gamma- x_f^2)^2 + 4b \tau^2 }} {2b}
\end{equation*}

\begin{figure}[h!]
	\centering
	\includegraphics[width=\textwidth, height=8cm]{GHO08a_fhn_local.eps}
		\caption{The local dynamics of one single isolated node with the parameters $\alpha = 1.05$, $\gamma=1.0$, $b=0.2$ and $\tau = 1.25$, the system is in quiescent state, the initial point given by red dot is chosen randomly. The fixed point is $[0.91, 0.66]$}
\end{figure}

The time evolution of $x$ and $y$ starts with rapid increase but both decay over time with small amplitude oscillations and even become constant over longer time. Activator and inhibitor relaxation onto the fixed point states that the system is at rest. 
\newpage

\subsubsection{Global Dynamics with Mutual Coupling and Time Delays}

For simplicity, we analyze the effect of mutual coupling only in two nodes system first. 

\begin{subequations}
 \begin{align}\dot{x_1} = \tau (y_1 + \gamma x_1 - \frac{x_1^3}{3}) + C [x_2(t-\tau_2^C)-x_1(t)] \label{eqn: frobenius 19}\\  \dot{y_1} = -\frac{1}{\tau} (x_1 - \alpha + b y_1) \label{eqn: frobenius 20} \\ \dot{x_2}=\tau (y_2 + \gamma x_2 - \frac{x_2^3}{3}) + C [x_1(t-\tau_1^C)-x_2(t)] \label{eqn: frobenius 21} \\  \dot{y_2} = -\frac{1}{\tau} (x_2 - \alpha + by_2) \label{eqn: frobenius 22}\end{align} 
\end{subequations}

where $C$ is the mutual coupling constant, $\tau^C$ is the time delays of mutual coupling, and subindices $1$ and $2$ are for two nodes. 


\begin{figure}[h!]
	\centering
	\includegraphics[width=\textwidth, height=8cm]{GHO08a_fhn_global.eps}
		\caption{The global dynamics of two nodes with the parameters $\alpha = 1.05$, $\gamma=1.9$, $b=0.2$, $\tau = 1.25$, $C=3$ and mutual coupling delays $\tau_1^C=\tau_2^C=\tau^C=5.0$.}
\end{figure}

\begin{figure}[h!]
	\centering
	\includegraphics[width=\textwidth, height=8cm]{GHO08_global_dynamics_C.eps}
		\caption{The global dynamics of two nodes with the parameters $\alpha = 0.89$, $\gamma=0.9$, $b=0.2$, $\tau = 4$, $C=5$ and mutual coupling delays $\tau_1^C=\tau_2^C=\tau^C=5.0$. Oscillatory behavior of activator and inhibitor due to coupling is observed.}
\end{figure}

\newpage

\section{FHN Model in [VUK13]} 

\subsection{Local Dynamics}
\begin{subequations}
 \begin{align}\dot{x} = \tau (y + \gamma x - \frac{x^3}{3})  \label{eqn: frobenius 17}\\  \dot{y} = -\frac{1}{\tau} (x - \alpha + b y - I ) \label{eqn: frobenius 18}   \end{align} 
\end{subequations}

\begin{figure}[h!]
	\centering
	\includegraphics[width=\textwidth, height=5.5cm]{VUK13_local_dynamics.eps}
		\caption{The local dynamics of two nodes with the parameters $\alpha = 0.85$, $\gamma=1.0$, $b=0.2$, $\tau = 1.4$ and $I=0$. Intersection of nullclines $[x_0, y_0] = (0.98,  -0.67)$. }
\end{figure}

The time evolution of $x$ and $y$ resembles damped oscillatory behavior. The system is stable, $x$ and $y$ relax on the fixed point over time as seen in state space.


\subsection{Global Dynamics}

\begin{subequations}
 \begin{align}\dot{x_1} = \tau (y_1 + \gamma x_1 - \frac{x^3}{3}) + C [x_2(t-\tau_2^C)-x_1(t)]  \label{eqn: frobenius 17}\\  \dot{y_1} = -\frac{1}{\tau} (x_1 - \alpha + b y_1 - I ) \label{eqn: frobenius 18} \\ \dot{x_2} = \tau (y_2 + \gamma x_2 - \frac{x_2^3}{3}) + C [x_1(t-\tau_1^C)-x_2(t)] \label{eqn: frobenius 18} \\  \dot{y_2} = -\frac{1}{\tau} (x_2 - \alpha + b y_2 - I ) \end{align} 
\end{subequations}

\begin{figure}[h!]
	\centering
	\includegraphics[width=\textwidth, height=7cm]{VUK13_global_dynamics.eps}
		\caption{The global dynamics of two nodes with the parameters $\alpha = 0.85$, $\gamma=1.0$, $b=0.2$, $\tau = 1.4$. System resembles a damped oscillatory behavior.}
\end{figure}

\section{FHN Time Series}

\begin{subequations}
 \begin{align}\dot{x_i} = \tau (y_i + \gamma x_i - \frac{x_i^3}{3}) -c \sum_{j=1}^N f_{ij}x_j(t - \Delta t_{ij}) +n_x \label{eqn: frobenius 17}\\  \dot{y_i} = -\frac{1}{\tau} (x_i - \alpha + b y_i - I ) +n_y \label{eqn: frobenius 18}   \end{align} 
\end{subequations}

where $c$ is coupling strength, $f_{ij}$ is the connectivity matrix, $\Delta t_{ij}$ is time delay due to finite signal propagation velocity between nodes, $n_u$ is the noise factor. $\Delta t_{ij}$ is calculated as $\Delta t_{ij}=\frac{d_{ij}}{\nu}$, distance matrix divided by velocity. 

\newpage

\subsection{Local Dynamics}

\begin{subequations}
 \begin{align}\dot{x} = \tau (y + \gamma x - \frac{x^3}{3})  \label{eqn: frobenius 17}\\  \dot{y} = -\frac{1}{\tau} (x - \alpha + b y - I ) \label{eqn: frobenius 18}   \end{align} 
\end{subequations}

\begin{figure}[h!]
	\centering
	\includegraphics[width=0.7\textwidth, height=6cm]{simple_graph.eps}
		\caption{A simple graph with four nodes, there is no connection among the nodes.}
\end{figure}


\begin{figure}[h!]
	\centering
	\includegraphics[width=\textwidth, height=7cm]{FHN_local_dynamics.eps}
		\caption{The activator-inhibitor time series of any node chosen from the simple graph with a randomly chosen initial point seen as red dot.}
\end{figure}

\newpage

\subsection{Global Dynamics}

\begin{subequations}
 \begin{align}\dot{x_i} = \tau (y_i + \gamma x_i - \frac{x_i^3}{3}) -c \sum_{j=1}^N f_{ij}x_j(t - \Delta t_{ij}) +n_x \label{eqn: frobenius 17}\\  \dot{y_i} = -\frac{1}{\tau} (x_i - \alpha + b y_i - I ) +n_y \label{eqn: frobenius 18}   \end{align} 
\end{subequations}


\begin{figure}[h!]
	\centering
	\includegraphics[width=0.7\textwidth, height=6cm]{A_small_graph.eps}
		\caption{A simple graph with four nodes, this time with connected nodes.}
\end{figure}

\begin{figure}[h!]
	\centering
	\includegraphics[width=0.9\textwidth, height=7cm]{FHN_time_series_real.eps}
		\caption{The activator-inhibitor time series of two nodes chosen from the simple connected graph. The noise ($n_x$, $n_y$) here is a Gaussian white noise distribution, the strength of noise is $D=0.05$, large noise. Differential equation set is solved numerically with PYTHON-module PYDELAY for $t=100s$. The propagation velocity is fixed to $v=7m/s$ for simplicity now to calculate the time delays among the nodes, $\Delta t_{ij} = d_{ij} /v$. The periodic oscillations seem to occur due to the mutual couplings amonf the nodes.}
\end{figure}
%
%\section{An Overlook to the Adaptive Solution and the Sample Solution of Time Delayed Differental Equations}
%
%In this section I would like to simulate FHN model given in [VUK13] for two simple nodes $i,j={1,2}$ with given distance matrix $d_{ij}$ via adaptive step size method and sampling solution method. The model for the neural dynamics in [VUK13] is expressed as below.
%
%\begin{equation}
% \dot{u}_i=g(u_i,v_i)-c \sum_{j=1}^N  f_{ij} u_j(t-\Delta t_{ij})+n_u
%\end{equation}
%\begin{equation}
% \dot{v}_i=h(u_i,v_i)+n_v
%\end{equation}
%
%where $c$ is coupling strength, $f_{ij}$ is the connectivity matrix, $\Delta t_{ij}$ is time delay due to finite signal propagation velocity between nodes, $n_u$ is the noise factor. $\Delta t_{ij}$ is calculated as $\Delta t_{ij}=\frac{d_{ij}}{\nu}$, distance matrix divided by velocity. 
%
%
%The functions $g$ and $v$ are modeled very similar to FitzHugh-Nagumo model introduced in [GHO08]:
%\begin{equation}
% \dot{u}=g(u,v)=\tau(v+\gamma u - \frac{u^3}{3})
%\end{equation}
%\begin{equation}
% \dot{v}=h(u,v)=-\frac{1}{\tau}(u- \alpha +bv-I)
%\end{equation}
%
%where $I$ is magnitude of an external stimulus, which is assumed to be 0. 
%
%The connectivity and distance matrices only for two nodes are introduced as the following;
%
%\[
%\textbf{$f_{ij}$}=
%\left[ {\begin{array}{cc }
%0.0000  &  5,6731.10^{-1} \\
%5,6731.10^{-1} &  0.0000  \\
%
%\end{array} } \right]
%\]
%
%
%
%\[
%\textbf{$d_{ij}$}=
%\left[ {\begin{array}{cc }
%0.0000  &  85,5102 \\
%85,5102 &  0.0000  \\
%
%\end{array} } \right]
%\]
%
%The purpose is to run our simulation $(fhn\_time\_series.py)$ to investigate the sampling rate effect on the reliability of data by comparing it to the real numerical solution method, namely adaptive step size method.
%
%\begin{figure}[h!]
%	\centering
%	\includegraphics[width=0.7\textwidth, height=9cm]{fhn_adapti.eps}
%		\caption{The simulation on top has $dt=0.1$ sampling rate and it lies perfectly on the numerical solutions marked by the dots. The sampling rate for the simulation below is $dt=1$, and it does not match well with the numerical solution.}
%\end{figure}
%
%\subsection{Simulation of FHN time series in [VUK13]}
%
%The bigger picture of [VUK13] notations..
%
%\begin{figure}[h!]
%	\centering
%	\includegraphics[width=\textwidth, height=9cm]{FHN_time_series_C_small.eps}
%		\caption{Sampling rate is $dt=0.1$, and $C=0.01$}
%\end{figure}
%
%\begin{figure}[h!]
%	\centering
%	\includegraphics[width=\textwidth, height=9cm]{FHN_time_series_C_big.eps}
%		\caption{Sampling rate is $dt=0.1$, and $C=4$}
%\end{figure}

\end{document}