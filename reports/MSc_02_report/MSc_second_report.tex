\section{FitzHugh-Nagumo Model Dynamics in [PAN12] and [GHO08]}

Section 2 analyzes the FitzHugh-Nagumo dynamics with two different models in [PAN12] and [GHO08] papers as isolated and coupled versions of attractor and inhibitor. The local dynamics in both papers are initially presented, then the effects of possible couplings are analyzed. 

\subsection{FHN Model in [PAN12]}

\subsubsection{Local Dynamics}

This assignment investigates the local dynamics of the FHN model in [PAN12]. Neither of the activator and inhibitor variables includes any coupling parameter for the simplicity at the beginning. We can consider a single node having activator and inhibitor variables isolated from other nodes in the bigger network. 
   
\begin{subequations} \begin{align} \varepsilon  \dot{x} = x - \frac{x^3}{3} -y   \label{eqn: frobenius 6}\\  \dot{y} = x+a \label{eqn: frobenius 7}   \end{align} 
\end{subequations}

where $x$ is activator variable, $y$ is the inhibitor variable, $\varepsilon$ denotes the time constant accelerating $x$ , and $a$ is the threshold parameter. 


\begin{figure}[h!]
	\centering
	\includegraphics[width=\textwidth]{PAN12_local_dynamics.eps}
	\caption{The local dynamics with the parameters $a=1.3$ and $\varepsilon = 0.01$. The time evolution shows a rapid excitation and less stronger rapid inhibition at very beginning, later on both $x$ and $y$ stays constant. The state space plot on the right shows the pattern of $x$ and $y$ together with the nullclines. $x$ values evolve faster than $y$.}
\end{figure}

\subsubsection{Global Dynamics with Mutual Coupling and Time Delays}

Now we can consider two nodes influencing each other through a mutual coupling and time delay. The nodes are not isolated from the bigger network.

\begin{subequations} \begin{align} \varepsilon  \dot{x_1} = x_1 - \frac{x_1^3}{3} -y_1  + C [x_2(t-\tau_2^C)-x_1(t)] \label{eqn: frobenius 10}\\  \dot{y_1} = x_1+a \label{eqn: frobenius 11}  \\ \varepsilon  \dot{x_2} = x_2 - \frac{x_2^3}{3} -y_2  + C [x_1(t-\tau_1^C)-x_2(t)] \label{eqn: frobenius 12}  \\  \dot{y_2} = x_2+a \label{eqn: frobenius 13} 
\end{align} 
\end{subequations}

where $C$ is the mutual coupling constant, $\tau^C$ is the time delays, and subindices $1$ and $2$ are for two nodes. The time delays must not be necessarily same, however, this assignment keeps both $\tau^C$ at the same value.

\begin{figure}[h!]
	\centering
	\includegraphics[width=\textwidth]{PAN12_global_dynamics_C.eps}
		\caption{The global dynamics of the two nodes with the parameters $a=1.3$ and $\varepsilon = 0.01$, $C=0.5$ and $\tau^C= 3.0$.}
\end{figure}

 The time evolution shows a rapid excitatory and less stronger rapid inhibitory behaviors of the nodes as the subsequent oscillation over time. The mutual coupling ($C$) let the $x$ and $y$ variables to oscillate instead of decaying on constant values as in Figure 15. The state space plot indicates a round trip over the nullclines, the change in $x$ is faster than $y$ as seen in excitation and inhibition peaks on time evolution plot.
 
\subsubsection{Global Dynamics with Mutual Coupling, Self Coupling and Time Delays}

This part investigates the complete FHN model given in [PAN12].

\begin{subequations} \begin{align} \varepsilon  \dot{x_1} = x_1 - \frac{x_1^3}{3} -y_1  + C [x_2(t-\tau_2^C)-x_1(t)] + K(x_1(t-\tau_1^K) - x_1(t)) \label{eqn: frobenius 14}\\  \dot{y_1} = x_1+a \label{eqn: frobenius 15}  \\ \varepsilon  \dot{x_2} = x_2 - \frac{x_2^3}{3} -y_2  + C [x_1(t-\tau_1^C)-x_2(t)] + K(x_2(t-\tau_1^K) - x_2(t)) \label{eqn: frobenius 16}  \\  \dot{y_2} = x_2+a \label{eqn: frobenius 17} 
\end{align} 
\end{subequations}

where $K$ is the strength of self coupling of the nodes and $\tau^K$ values are the time delays required for the self coupling, both nodes are assumed to have both self coupling delays for simplicity.

\begin{figure}[h!]
	\centering
	\includegraphics[width=\textwidth, height=8cm]{PAN12_global_dynamics_C_K.eps}
		\caption{The global dynamics of the two nodes with the parameters $a=1.3$ and $\varepsilon = 0.01$, $C=0.5$, $\tau^C= 3.0$, $K=0.1$, $\tau^K= 3.0$.}
\end{figure}

The mutual coupling constitutes the oscillatory behavior of excitation and inhibition at a higher frequency as seen in time evolution plot in Figure 17 compared to Figure 16. The phase space changes only slightly.

\subsection{FHN Model in [GHO08]}

\subsubsection{Local Dynamics}

\begin{subequations}
 \begin{align}\dot{x} = \tau (y + \gamma x - \frac{x^3}{3})  \label{eqn: frobenius 17}\\  \dot{y} = -\frac{1}{\tau} (x - \alpha + b y) \label{eqn: frobenius 18}   \end{align} 
\end{subequations}

where $x$ is activator variable, $y$ is the inhibitor variable, $\tau$ denotes the time constant accelerating $x$, and $\gamma$, $\alpha$, $b$ are the parameters. 

\begin{figure}[h!]
	\centering
	\includegraphics[width=\textwidth, height=8cm]{GHO08_local_dynamics.eps}
		\caption{The local dynamics of one single isolated node with the parameters $\alpha = 0.89$, $\gamma=0.9$, $b=0.1$ and $\tau = 4$.}
\end{figure}

The time evolution of $x$ and $y$ starts with rapid increase but both decay over time with small amplituded oscillations and even become constant over longer time. 

\subsubsection{Global Dynamics with Mutual Coupling and Time Delays}

\begin{subequations}
 \begin{align}\dot{x_1} = \tau (y_1 + \gamma x_1 - \frac{x_1^3}{3}) + C [x_2(t-\tau_2^C)-x_1(t)] \label{eqn: frobenius 19}\\  \dot{y_1} = -\frac{1}{\tau} (x_1 - \alpha + b y_1) \label{eqn: frobenius 20} \\ \dot{x_2}=tau (y_2 + \gamma x_2 - \frac{x_2^3}{3}) + C [x_1(t-\tau_2^C)-x_2(t)] \label{eqn: frobenius 21} \\  \dot{y_2} = -\frac{1}{\tau} (x_2 - \alpha + by_2) \label{eqn: frobenius 22}\end{align} 
\end{subequations}

where $C$ is the mutual coupling constant, $\tau^C$ is the time delays of mutual coupling, and subindices $1$ and $2$ are for two nodes. 

\newpage

\begin{figure}[h!]
	\centering
	\includegraphics[width=\textwidth, height=8cm]{GHO08_global_dynamics_C.eps}
		\caption{The global dynamics of two nodes with the parameters $\alpha = 0.89$, $\gamma=0.9$, $b=0.1$, $\tau = 4$, $C=5$ and mutual coupling delays $\tau_1^C=\tau_2^C=\tau^C=3$.}
\end{figure}

The non-zero coupling strength initiates the oscillatory behavior of $x$ and $y$ now. 

\subsubsection{Global Dynamics with Mutual Coupling, Self Coupling and Time Delays}

\begin{subequations}
 \begin{align}\dot{x_1} = \tau (y_1 + \gamma x_1 - \frac{x_1^3}{3}) + C [x_2(t-\tau_2^C)-x_1(t)] + K(x_1(t-\tau_1^K) - x_1(t)) \label{eqn: frobenius 19}\\  \dot{y_1} = -\frac{1}{\tau} (x_1 - \alpha + b y_1) \label{eqn: frobenius 20} \\ \dot{x_2}=tau (y_2 + \gamma x_2 - \frac{x_2^3}{3}) + C [x_1(t-\tau_2^C)-x_2(t)]+ K(x_2(t-\tau_2^K) - x_2(t)) \label{eqn: frobenius 21} \\  \dot{y_2} = -\frac{1}{\tau} (x_2 - \alpha + by_2) \label{eqn: frobenius 22}\end{align} 
\end{subequations}

\newpage

\begin{figure}[h!]
	\centering
	\includegraphics[width=1\textwidth, height=9cm]{GHO08_global_dynamics_C_K.eps}
		\caption{The global dynamics of two nodes with the parameters $\alpha = 0.89$, $\gamma=0.9$, $b=0.1$, $\tau = 4$, $C=5$, mutual coupling delays $\tau_1^C=\tau_2^C=\tau^C=3$, $K=4$, and self coupling delays $\tau_1^K=\tau_2^K=\tau^K=3$ .}
\end{figure}

The mutual coupling adds up new $x$ and $y$ oscillations to the system.

\section{An Overlook to the Adaptive Solution and the Sample Solution of Time Delayed Differental Equations}

In this section I would like to simulate FHN model given in [VUK13] for two simple nodes $i,j={1,2}$ with given distance matrix $d_{ij}$ via adaptive step size method and sampling solution method. The model for the neural dynamics in [VUK13] is expressed as below.

\begin{equation}
 \dot{u}_i=g(u_i,v_i)-c \sum_{j=1}^N  f_{ij} u_j(t-\Delta t_{ij})+n_u
\end{equation}
\begin{equation}
 \dot{v}_i=h(u_i,v_i)+n_v
\end{equation}

where $c$ is coupling strength, $f_{ij}$ is the connectivity matrix, $\Delta t_{ij}$ is time delay due to finite signal propagation velocity between nodes, $n_u$ is the noise factor. $\Delta t_{ij}$ is calculated as $\Delta t_{ij}=\frac{d_{ij}}{\nu}$, distance matrix divided by velocity. 


The functions $g$ and $v$ are modeled very similar to FitzHugh-Nagumo model introduced in [GHO08]:
\begin{equation}
 \dot{u}=g(u,v)=\tau(v+\gamma u - \frac{u^3}{3})
\end{equation}
\begin{equation}
 \dot{v}=h(u,v)=-\frac{1}{\tau}(u- \alpha +bv-I)
\end{equation}

where $I$ is magnitude of an external stimulus, which is assumed to be 0. 

The connectivity and distance matrices only for two nodes are introduced as the following;

\[
\textbf{$f_{ij}$}=
\left[ {\begin{array}{cc }
0.0000  &  5,6731.10^{-1} \\
5,6731.10^{-1} &  0.0000  \\

\end{array} } \right]
\]



\[
\textbf{$d_{ij}$}=
\left[ {\begin{array}{cc }
0.0000  &  85,5102 \\
85,5102 &  0.0000  \\

\end{array} } \right]
\]

The purpose is to run our simulation $(fhn\_time\_series.py)$ to investigate the sampling rate effect on the reliability of data by comparing it to the real numerical solution method, namely adaptive step size method.

\begin{figure}[h!]
	\centering
	\includegraphics[width=0.7\textwidth, height=9cm]{fhn_adapti.eps}
		\caption{The simulation on top has $dt=0.1$ sampling rate and it lies perfectly on the numerical solutions marked by the dots. The sampling rate for the simulation below is $dt=1$, and it does not match well with the numerical solution.}
\end{figure}

\subsection{Simulation of FHN time series in [VUK13]}

The bigger picture of [VUK13] notations..

\begin{figure}[h!]
	\centering
	\includegraphics[width=\textwidth, height=9cm]{FHN_time_series_C_small.eps}
		\caption{Sampling rate is $dt=0.1$, and $C=0.01$}
\end{figure}

\begin{figure}[h!]
	\centering
	\includegraphics[width=\textwidth, height=9cm]{FHN_time_series_C_big.eps}
		\caption{Sampling rate is $dt=0.1$, and $C=4$}
\end{figure}