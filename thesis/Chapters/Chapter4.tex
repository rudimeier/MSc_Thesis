% Chapter 4

\chapter{Conclusion and Discussion}  % Main chapter title

\label{Chapter4} % For referencing the chapter elsewhere, use \ref{Chapter1} 

\lhead{Chapter 4. \emph{Conclusion and Discussion}} % This is for the header on each page - perhaps a shortened title

%----------------------------------------------------------------------------------------
The project utilizes modeling approaches combined with empirical results to resolve underlying biophysical mechanisms of human brain at resting state. Empirical brain connectivity maps of resting state are obtained from fMRI-BOLD and DW-MRI techniques, revealing functional and anatomical connections among AAL regions, respectively. The modeling approaches are implemented to discover \textit{i)} neuronal activity time-series, \textit{ii)} ultra-slow BOLD fluctuations, and \textit{iii)} to investigate topological properties of brain graphs. Temporal dynamics of neuronal populations is built on FitzHugh-Nagumo (FHN) oscillations \citep{GHO08, VUK13, DEC09, FIT61}. The BOLD activity is inferred via the Balloon-Windkessel hemodynamic model, which takes the normalized FHN time-series as an input \citep{FRI00, VUK13}. The spatial properties of brain graphs are discussed by comparing network measures of brain graphs to randomly constructed graphs with statistical methods \citep{BUL09, RUB09, NEW10}. The research proposal  is capture temporal fMRI-BOLD dynamics through structural connectivity map of brain, while discussing if the spatial topology properties of brain networks are distinguishable than that of random graphs. 

The fMRI-BOLD functional correlation matrix can be recaptured with FHN modeled neuronal activity dynamics at high axonal signal propagation velocity $v>6$ m/s, at intermediate coupling strength $0.1<c<0.4$ and at threshold $0.54<r<0.60$. These parameter ranges are in agreement with previous studies \citep{VUK13, GHO08a}. It is possible to follow traces of highly correlating AAL regions located symmetrically on right and left hemispheres as presented with sub-diagonals in Figure 3.2, left.  

The DW-MRI anatomical correlation matrix can also be imitated with FHN model applied on ACM based brain graphs at $v>4$ m/s, at $0.1<c<0.5$, and at connection probability range $0.18<p<0.70$. The symmetry of empirical correlation matrix is preserved in simulated correlation matrix as seen in Figure 3.6.

Brain graphs are constructed on adjacency matrices, which are binarized empirical FCMs and ACMs via $r$ and $p$, respectively. Here, $r$- and $p$-values yield us to identify network topology of simulated brain graphs, i.e.  $0.54<r<0.60$ corresponds to a network density $0.50< \kappa_{FCM} <0.18$ for FCM graphs, and $0.18<p<0.70$ that of $0.30< \kappa_{ACM} <0.18$ for ACM graphs. The lower boundaries present the limit of \textsc{PYDELAY} module for the numerical solution of time-delayed differential equations of FHN model. The upper limits are restrained with statistical characterizations of brain graphs. Beyond $\kappa_{FCM} <0.18$ and $\kappa_{ACM}<0.18$, less densely connected brain networks exhibit dramatically changing transitivity $T$, shortest pathway $d_{ij}$, small worldness $S$ and assortativity $A$, and simulations become distinctly different from experiment (See Section 2.4 and Appendix B).

One of the key proposals of this master's thesis is to investigate whether it is possible to catch BOLD fluctuations through structural connections in the human brain at resting state. FHN neuronal activity model is promising, but it is not a complete approach for BOLD dynamics due to high frequency oscillations $20$ Hz $< \nu <60 $ Hz of type-II excitable neuronal populations. The inferred Balloon-Windkessel model provides high correlations between FCM based BOLD simulations and fMRI-BOLD data at $c<0.1$, stating that inferred BOLD activity model is plausible at least for the less strongly functionally coupled neurons. However, the principal step is to capture BOLD fluctuations via ACM brain graphs for the thesis proposal. The correlation between ACM based BOLD simulations and fMRI-BOLD is found to be restricted by a Pearson correlation coefficient of $\rho_{e,s}=0.22$ on parameter space $(p,c)$ (Figure 3.9 and 3.12). The coupling strength range is $c<0.1$ for high correlations. Oppositely to FHN neuronal activity simulations, the BOLD fluctuations become more reasonable at very small $c$ for ACM and FCM graphs. $c$ scales the amplitude of neuronal activity oscillations, so, small scaled FHN  oscillating neurons turn out better BOLD activity simulations. The parameter analysis of simulated activity of BOLD signal for ACM graphs can be designed with finer $c$ values with this deduction in future. 

The intrinsic properties of a brain network have a significant effect on its temporal dynamics. This proposal is evidenced statistically, when the modeled neuronal activities of nodes in the brain graph is compared to that of random graphs. The brain graphs are found to be distinguishable than random graphs at specific parameter ranges, i.e. at low coupling strength, in terms of extracted FHN time-series. 
 

 
